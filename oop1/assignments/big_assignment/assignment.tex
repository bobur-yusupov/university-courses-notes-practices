\documentclass[a4paper,12pt]{article}
\usepackage[utf8]{inputenc}
\usepackage{amsmath, amssymb}
\usepackage{graphicx}
\usepackage{float}
\usepackage{tikz}
\usepackage{listings}
\usepackage{xcolor}
\usepackage{caption}
\usepackage{geometry}
\geometry{margin=0.5in}

\title{Big Assignment: Object-Oriented Programming\\\large Design of Project}
\author{Yusupov Boburjon\\Neptun Code: YTAJDI}
\date{}

\begin{document}

\maketitle

\newpage

\section*{8. Wildlife Simulation}

Wildlife conservation is the most critical issue in a major game reserve.  
Trained rangers venture out daily to spot wildlife, administer medicine to diseased animals, and protect the habitat from desecrating poachers. The reserve is a well-organized company where rangers, animals, habitats, and support vehicles coexist in symbiosis for fostering nature’s fragile balance.

All animals have a species, health, and a stress level. Animals become stressed or injured due to natural events or poachers. The rangers can calm them down and treat them.

Rangers are diligent workers who protect the reserve’s environment. Every ranger has a name, experience, and efficiency. They can assist an animal by improving its status or reducing stress. They can also detect animals that need help among all the animals and help them. If they are in the same habitat as a poacher, they find and fight the poacher. The winner is determined based on the strength of both the ranger and the poacher. The strength is initialized randomly.

Poachers are intruders who hunt down animals. Each poacher targets a specific species. Different poachers are dangerous to a different degree. Poachers disturb wildlife and cause them stress as they evade their hunters. They can also hurt the animals if they catch them.

Each habitat has a name, a capacity, and the animals living in it. Animals can enter and leave habitats. If there is an incursion by a poacher, it stresses all animals in the habitat.

The rangers travel in Jeep vehicles. The vehicle has an ID, fuel level, and a capacity. Vehicles can be refueled and deployed to specific habitats.

The simulation is run day-by-day. Each day, animals wander between habitats, rangers are sent to habitats to patrol them, and poachers go to habitats to hunt animals. The simulation reports at the end of the day the state of the reserve and the happenings on that day — for example, what animals were saved.

Play it out, roll out your rangers, and save the wildlife!

\bigskip

\textbf{Input file examples}

\textbf{\texttt{animals.txt}}
\begin{itemize}
    \item Format
    \newline \text{Species Health StressLevel}
    \item Example
    \begin{verbatim}
        Elephant 60 30
        Giraffe 70 20
        Zebra 80 25
    \end{verbatim}
\end{itemize}

\textbf{\texttt{rangers.txt}}
\begin{itemize}
    \item Format
    \newline \text{Name Experience Efficiency}
    \item Example
    \begin{verbatim}
        Balazs 5 7
        Sofiia 6 6
    \end{verbatim}
\end{itemize}

\textbf{\texttt{vehicles.txt}}
\begin{itemize}
    \item Format
    \newline \text{ID FuelLevel Capacity}
    \item Example
    \begin{verbatim}
        V1 100 2
        V2 100 3
    \end{verbatim}
\end{itemize}

\textbf{\texttt{habitats.txt}}
\begin{itemize}
    \item Format
    \newline \text{Name Capacity}
    \item Example
    \begin{verbatim}
        Savannah 3
    \end{verbatim}
\end{itemize}

\end{document}
